\section{Previous work}

My work mainly builds upon Edwin Brady's and Kevin Hammond's work on managing
concurrent resources in Idris~\cite{cbconc-fi} and Brady's Effects
library~\cite{effects-idr} for programming using algebraic effects in Idris.

\subsection{A concurrent resource management eDSL}

In the ``Correct-by-Construction Concurrency'' (CbC)~\cite{cbconc-fi} paper, an
embedded domain specific language for managing concurrent resources using
ordered locks is implemented, which ensures thread-safe and deadlock-free usage
of resources, which is formally proved in the paper.

The implementation of our effect closely follows the way shared resource usage
was modelled in the CbC paper.

\subsection{Algebraic Effects}

Algebraic effects is a relatively new approach to handling computation with
effects. From a pragmatic point of view, one could see algebraic effects as a
slightly simplified and more useful approach to model effectful computation than
monads. This is because of two reasons: a single algebraic effect can have many
\emph{handlers} which determine the actual execution of the effect, hence
making effects very re-usable and modular. Moreover, algebraic effects compose
more easily, unlike monads where one has to often take explicit care of the
ordering and use monad transformers to deal with non-trivial compositions.

For more information on algebraic effects we refer the reader to Andrej Bauer's
and Matija Pretnar's implementation of \texttt{Eff}~\cite{eff} and also Edwin Brady's work~\cite{effects-idr}.

In this paper we use the algebraic effects as they are implemented in the
Effects library of Idris.
