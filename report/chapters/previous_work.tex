\section{Previous work}

Our work mainly builds upon Edwin Brady's and Kevin Hammond's work on managing
concurrent resources in Idris~\cite{cbconc-fi} and Brady's Effects
library~\cite{effects-idr} for programming using algebraic effects in Idris.

\subsection{A concurrent resource management eDSL}

In the ``Correct-by-Construction Concurrency'' (CbC)~\cite{cbconc-fi} paper, an
embedded domain specific language for managing concurrent resources using
ordered locks is implemented. The eDSL ensures thread-safe and deadlock-free
usage of resources, which is formally proved in the paper.

The implementation of our effect closely follows the way shared resource usage
was modelled in the CbC paper.

\subsection{Algebraic Effects}

Algebraic effects are a relatively new approach to handling computation with
side-effects. From a pragmatic point of view, one could see algebraic effects
as a more modular and convenient approach to managing effectful code than
monads. Firstly this is because a single algebraic effect can have many
\emph{handlers} which determine the actual execution of the effect, hence
making effects highly re-usable and modular. Secondly, algebraic effects
compose naturally, unlike monads where one has to often take explicit care of
the ordering and use monad transformers for composition.

The most basic example presented in Brady's paper~\cite{effects-idr}
is that of \code{STATE} effect, which is essentially the same as the familiar
\code{State} monad in Haskell.

The \code{STATE} effect supports two operations, \code{get} and \code{put}:

\begin{BVerbatim}

get : Eff m [STATE a] a
put : a -> Eff m [STATE a] ()

\end{BVerbatim}

This should be read as: \code{get} is an effectful operation of \code{STATE a}
(where \code{a} is the type of the state) which can be executed in any
computational context \code{m} and produces an output of type \code{a}.

For simplicity the computational context can be assumed to be a monad.

Similarly, \code{put} says that it takes an input of type \code{a}, and
performs an effectful action on \code{STATE} holding a type \code{a}, producing
no output (the output type is a unit).

The execution of these operations is then determined by the computation context
in which they are executed. E.g. one could take the context to be a
\code{State} monad and write handlers which execute \code{get} and \code{put}
as the corresponding operations in the \code{State} monad. Alternatively, one
could write a handler which uses persistent storage for this.

In the Effects library, effectful operations are defined algebraically as data
types and handlers are defined by providing instances for a \code{Handler}
typeclass which is parametrised over an effect and a computational context.

For more information on algebraic effects we refer the reader to Andrej Bauer's
and Matija Pretnar's implementation of \texttt{Eff}~\cite{eff} and also Edwin
Brady's work~\cite{effects-idr}.

In this paper we use the algebraic effects as they are implemented in the
Effects library of Idris.
