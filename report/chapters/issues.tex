\section{Issues}

We now discuss the two main issues with our implementation of the
\code{ConcState} effect.

\subsection{Proof terms}

The cost of enforcing safe concurrent access at the type level is the increase
in complexity of the types signatures. In particular, as we have seen in the
examples, the user (i.e.~the programmer) of the effect is still expected to
provide proofs of resource membership.

What's worse is that non-trivial combinations of the \code{ConcState} actions
are sufficient to confuse the type checker, hence it becomes hard to write
generic functions (such as the {\code{bump\_ith\_val}), because it also
involes manual proving.

It would be quite easy to add proof automation which would resolve at least
most of the trivial cases (such as plain membership). However, the proof
automation could only work as long as it is able to keep up with the complexity
of the ways in which resource access is composed. This is very important
e.g.~in a database or an OS setting where lock usage can become extremely
complicated, yet it is crucial to provide safety and liveliness guarantees.

\subsection{Composition with other effects}

Finally, we come back to the promised discussion of problems related to
\code{Fork}. So far our examples were simple enough so as not to reveal any
issues with the forking action.

However, consider what would happen if we tried to fork a program, which is
using some other effect together with \code{CONCSTATE}. As an example we
present this silly code:

\begin{BVerbatim}

hello_in_a_fork: Eff IO [CONCSTATE rsin IO, STDIO] ()
hello_in_a_fork = do
    _ <- efork (putStrLn "Hello")
    return ()

\end{BVerbatim}

Firstly, this code will fail to typecheck with the following error:

\begin{BVerbatim}

Can't unify
    Eff e [STDIO] ()
with
    Eff IO [CONCSTATE rsin IO] ()

Specifically:
    Can't unify
        ()
    with
        REnv rsin

\end{BVerbatim}

and it is obvious why this is so if we look again at the type of \code{efork}:

\begin{BVerbatim}

efork: {auto prf: allUnlocked rsin = True} ->
        (prog: Eff m [CONCSTATE rsin m] ()) ->
        Eff m [CONCSTATE rsin m] ()
efork prog {prf} = (Fork prf prog)

\end{BVerbatim}

We specified that the to-be-forked program only uses the \code{CONCSTATE}
effect, so the typecheker validly disagrees with the usage of \code{STDIO}.

It seems the ``fix'' for this is obvious: specify in the type that
\code{CONCSTATE} is simply an element of the available effects.

However this approach breaks down when we have to implement a handler for
\code{efork}. Consider again the handler code:

\begin{BVerbatim}

    handle env (Fork _ prog) k = do
        _ <- fork (run [env] prog)
        k env ()

\end{BVerbatim}

The problem here is that we \emph{only} have access to the environment
(\code{env}) of the respective effect that is being handled, i.e.
\code{ConcState}. However, in order to execute an effectful program with
multiple effects, we would need to have access to their respective environments
too.

Hence, at least using our straight-forward approach, it seems it is impossible
(in the current Effects library) to write a handler for \code{Fork} which would
accept effectful programs with multiple effects.

In fact, this is a known implementation issue described in the
Effects~\cite[p.~12]{effects-idr} paper:

\begin{quote}
\emph{
Another weakness which we hope to address is that mixing control and resource
effects is a challenge. For example, we cannot currently thread state through
all branches of a non-deterministic search. If we can address this, it may be
possible to represent more sophisticated effects such as co-operative
multithreading, or even partiality. One way to tackle this problem could be to
introduce a new method of the Handler class which manages resources more
precisely, with a default implementation calling handle.
}
\end{quote}

In the specific case of handling \code{Fork} it would be sufficient to have
access to all of the environments in the \code{handle} function, because this
would allow us to use the earlier described ``fix'' (changing the type of
\code{Fork}) and then writing something like:

\begin{BVerbatim}

    handle' all_environments (Fork _ prog) k = do
        _ <- fork (run all_environments prog)
        k env ()

\end{BVerbatim}

which would then allow us to fork \code{CONCSTATE} programs composed with any
other effects.
