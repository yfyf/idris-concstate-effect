\section{Conclusions and possible further work}

We have succeeded in (re-)implementing type-safe concurrent resource
management as an algebraic effect in Idris. Despite the issues discussed,
avoiding the need for a DSL is already a big step forward and shows that we can
express complicated notions using algebraic effects and dependent types quite
neatly. Further work obviously includes resolving the discussed issues.

The current proof tactics available in Idris should be sufficient to automate
most of the theorem proving required for the \code{ConcState} effect and would
be an interesting challenge to solve.

On the other hand, the problems we faced while implementing a handler for
\code{Fork} are due to the way the Effects library is currently implemented. It
would be interesting to explore potential improvements to the library. One
approach, as suggested by Brady, would be to introduce a more ``low-level''
\code{handle'} function for the \code{Handle} class, which would allow to
handle effects within the context of other effects.

This improvement, in addition to hopefully resolving the issues we faced, might
also enable splitting up the \code{ConcState} effect into smaller effects
where, e.g., the resources are standalone ``indexed'' \code{STATE}-like effects
belonging to some group (indexing and grouping would allow to maintain some
ordering) with the usual state actions and also actions for locking and forking.

An alternative might be to consider forking also as a standalone effect, which,
when handled, could require other effects to be in some (``safe'') states.
This however would not be straightforward as it potentially introduces
cyclic dependencies between different effects and their handlers.

\subsection*{Acknowledgments}

I would like to thank my supervisor Wouter Swierstra for his good-will,
inspiring comments and patience. I would also like to thank the developers and
contributors of Idris for their support.
